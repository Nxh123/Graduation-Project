% !TeX root = ../main.tex

% \chapter{简介}

% \section{一级节标题}

% \subsection{二级节标题}

% \subsubsection{三级节标题}

% \paragraph{四级节标题}

% \subparagraph{五级节标题}

% 本模板 \pkg{ustcthesis} 是中国科学技术大学本科生和研究生学位论文的 \LaTeX{}
% 模板, 按照《\href{http://gradschool.ustc.edu.cn/static/oldsite/ylb/material/xw/wdxz/32.pdf}
% {中国科学技术大学研究生学位论文撰写手册}》(以下简称《撰写手册》)和
% 《\href{https://www.teach.ustc.edu.cn/notice/notice-teaching/11530.html}
% {关于本科毕业论文(设计)格式和统一封面的通知}》的要求编写。

% Lorem ipsum dolor sit amet, consectetur adipiscing elit, sed do eiusmod tempor
% incididunt ut labore et dolore magna aliqua.
% Ut enim ad minim veniam, quis nostrud exercitation ullamco laboris nisi ut
% aliquip ex ea commodo consequat.
% Duis aute irure dolor in reprehenderit in voluptate velit esse cillum dolore eu
% fugiat nulla pariatur.
% Excepteur sint occaecat cupidatat non proident, sunt in culpa qui officia
% deserunt mollit anim id est laborum.



% \section{脚注}

% Lorem ipsum dolor sit amet, consectetur adipiscing elit, sed do eiusmod tempor
% incididunt ut labore et dolore magna aliqua.
% \footnote{Ut enim ad minim veniam, quis nostrud exercitation ullamco laboris
%   nisi ut aliquip ex ea commodo consequat.
%   Duis aute irure dolor in reprehenderit in voluptate velit esse cillum dolore
%   eu fugiat nulla pariatur.}
\chapter{介绍}
\section{研究背景}
由于数据资源的巨大经济价值潜力,近年来出现了许多在线数据交易系统\cite{jung2017accounttrade},
如CitizenMe、DataExchange、Datacoup、Factual、Terbine等,数据消费者可以通过
这些系统搜索和购买他们感兴趣的数据。然而,现实世界中的大部分数据都是由少数研究机构或
公司保存的,它们只是为了自己的分析目的,而不是与那些有数据需求但自己没有能力收集数
据的人分享,这导致交易系统中的数据量有限。由于很难从这些交易系统中获得适当的数据,因此极大
地抑制了用户使用这些系统的意愿。为了解决这个问题,众包数据交易的概念被提出。众包数据交易与
传统数据交易相比,在数据源获取方面具有更大的优势。它无需购置和部署专业的传感器节点,可以利
用用户的移动性和其携带的传感设备,在广泛的区域内收集具有经济价值的数据\cite{tong2020spatial,wang2017efficient,he2017exchange}。
一个典型的众包数据交易系统包括三个主要组成成分:数据卖家,数据买家和数据代理。根据数据买家给定
的需求,数据代理雇佣数据卖家来收集数据,然后将原始数据或经过处理的数据卖给相应的买家。然而,传统
的众包数据交易依赖于一个可信的第三方代理,这将导致用户怀疑交易的可靠性与真实性,从而降低用户
使用这种交易系统的积极性。

区块链技术的出现\cite{nakamoto2019bitcoin}为该问题提供了一种解决办法。由于其可靠性和透明性,
用户几乎不可能篡改区块链上的数据。使用区块链技术,互不信任的的用户可以在缺少可信第三方代理的情况
下完成可靠的数据交易和货币结算,同时这也避免了高昂的代理费用。为了完成基于区块链的去中心化的众包
数据交易,可以使用部署在区块链上的自动执行的程序,称为智能合约\cite{buterin2014next}。智能合约
可以强制用户履行各自的义务,从而代替可信第三方作为数据代理,完成去中心化的数据交易。
\section{研究目标}
设计一种基于区块链和不完全信息博弈的众包数据交易模型,能够不依赖于可信第三方,同时能够最大化参与
方的利益并保护数据版权。